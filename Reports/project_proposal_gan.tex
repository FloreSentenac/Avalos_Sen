\documentclass[11pt,a4paper]{article}
\usepackage[utf8]{inputenc}
\usepackage{amsmath}
\usepackage{amsfonts}
\usepackage{amssymb}
\usepackage{amsthm}
\usepackage{amsbsy}
\usepackage[
  margin=3.5cm,
  includefoot,
  footskip=30pt,
]{geometry}
\usepackage{layout}
\usepackage{graphicx}
\usepackage{listings}
\author{Flore Sentenac \\ flore.sentenac@gmail.com, \and Raphaël Avalos \\ raphael.avalos@telecom-paristech.fr}
\title{Image translation using generative adversarial networks (GANs)}
\date{}
\begin{document}
\maketitle
\section*{Motivation}
GANs (Generative Adversarial Networks) correspond to an approach for training and using convolutional neural networks as data generators, which have recently attracted a lot of attention in computer vision thanks to their ability to fit complicated image distributions and generate extremely realistic samples. Recently, GANs have been used for unsupervised/unpaired image translation producing convincing samples of a target distributions. Day-to-night image translation is one example demonstrating the power of GANs to capture image statistics.


\section*{Plan of work}
Our current plan of work for this project is the following:
\begin{itemize}
\item The first part of our work will be to understanding how GANs work:
\begin{itemize}
\item for modeling a particular image distribution, how they are trained and how they are used for sampling
\item for Image-to-Image translation in the supervised setting
\item for Image-to-Image translation in the unsupervised/unpaired setting
\end{itemize}
\item In the second part we will TODO
\end{itemize}

\end{document}